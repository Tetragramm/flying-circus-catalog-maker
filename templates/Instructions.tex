\documentclass{article}
\usepackage{graphicx}
\usepackage{hyperref}
\usepackage{xcolor}
\usepackage{csquotes}
\title{How to use the \LaTeX Catalog Assistant}
\author{Tetragramm}

\begin{document}
\maketitle
\section{Introduction}
The point of this tool is to make building aircraft catalogs in the style of the Flying Circus ones easier.  You will be able to make fancier things if you know and understand \LaTeX, but it is not necessary to use this tool.
\section{Why \LaTeX?}
\LaTeX is basically a programming language for laying out documents, whether academic papers or books or slides for presentations.  It is very powerful, but very complicated.  Additional features can be added by including packages, basically libraries of layout functions, that can make complicated things simpler.

For example, mirroring text is not something that can be done in a normal text editor, but in \LaTeX, it is simple.  The text \reflectbox{Mirror}|Mirror is made using the code \begin{verbatim}\reflectbox{Mirror}|Mirror\end{verbatim} using the package called "graphicx".

Only some packages are available, with the most common commands you may use described below in the section on aircraft descriptions.
\section{Setup}
First, you must install the dependencies.
\begin{itemize}
    \item \href{https://www.python.org/}{\color{blue}\underline{Python 3}}  (Ensure the box ``Add Python to environment variables'' is checked)
    \item \href{https://tug.org/texlive/}{\color{blue}\underline{TeX Live 2021}}
    \item The Fonts (provided)
\end{itemize}
Fortunately, these are nice simple installers, and you should be able to handle it fine.  The TeX Live installer does download a lot of packages, which can take some time.  For slower internet connections, potentially hours.  The fonts are stored in the folder called "fonts".  On Windows, just right click on them and hit install.
\section{What you Get}
In this folder are the fonts to install, some template files that are used during processing, and the Python scripts Create.py and Compile.py.  That's it!  That's all you need, besides your airplane designs.
\section{How to Use}
\begin{enumerate}
    \item Go to the \href{https://tetragramm.github.io/PlaneBuilder/index.html}{\color{blue}\underline{Plane Builder}} and load each of your aircraft designs.  Save them using the "Save Catalog" button at the bottom of the page, in this folder.
    \item Rename those files so they are in the order you wish them in the catalog.  I suggest naming them something like "A\_First\_Plane.txt", "B\_Second\_Plane.txt", ect.
    \item Open each of them and change the line "Insert Nickname Here" to the actual nickname of the plane.
    \item Open a command prompt or powershell window in the folder.  Hold the Shift key and right click in the folder, and click the ``Open command prompt window here'' or ``Open PowerShell window here'' options.
    \item Alternatively, do this by searching in the start menu, then using the command "cd" (change directory) to navigate to this folder.
    \item Run the command \begin{verbatim}
        python .\Create.py
    \end{verbatim} and when asked, type the Title and zero or more authors to include. When you are done entering authors, just press enter on an empty line.
    \item Wait just a minute as a gigantic pile of text cascades down the window.  Don't worry, it doesn't matter.
    \item Note the changes to the directory.  Three new folders (desc, images, subfiles), an AuthorInfo.text, some temporary files created by \LaTeX, and most importantly, a pdf file!
\end{enumerate}

\section{What to Change}
The file named AuthorInfo.text has the first line as the Title, and each following line is an author's name.  This way you don't need to re-enter those if you add additional planes and need to re-run Create.py

The folder "images" contains the aircraft images, or placeholders.  You can replace them with the images you want.  The name of the file is what is important.  For example you may replace "Basic\_Biplane\_image.png" with "Basic\_Biplane\_image.jpg" with no issues, but trying to use "Basic\_Biplane.png" will not work.  The images are automatically resized to fit, but you may wish to make or edit images into the preferred aspect ratio for better appearances.

The folder "desc" contains all of the text you edit.  For each plane there is a "PlaneName\_desc.txt" and a "PlaneName\_table.txt".

\subsection{The Table}
The table is the simpler file.  Each row consists of two parts, separated by an = sign, like the default one, reproduced below.
\begin{displayquote}
    Role=Edit, Add or\\
    Served With=remove lines\\
    First Flight=to fill\\
    Strengths=out the\\
    Weaknesses=table\\
    Inspiration=like this.\\
\end{displayquote}
The part to the left of the equal sign makes up the first column of the plane's table.  The part to the right, the second.  Don't put more than one equal sign per row, or it won't work.  You can add or remove rows to your heart's content, and are not limited to the ones already there, which were chosen because that's what the first catalog used.

\subsection{The Description}
The description file is actually a very simple \LaTeX file.  It has a two line header, and a one line footer.  In-between the \textbackslash begin and \textbackslash end is the place where you put your aircraft descriptions.  The text within it will be distributed evenly over the two columns of the page.

Because it is a \LaTeX document, you can easily use simple commands, and with a little effort, more complex formatting.  Check out the lovely tutorial at \href{https://overleaf.com/learn/latex/Paragraphs_and_new_lines}{\color{blue}\underline{Overleaf}} for how to do even complicated formatting.  For basic work, see the next section.
When you make changes, you will notice they don't show up in the PDF.  To see the results, you must compile the document.  If you have added, removed, re-ordered an airplane (Or changed values in the save file from the catalog), you will need to re-run the Create script.
\begin{verbatim}
    python .\Create.py
\end{verbatim}
It will read in everything and spit out a fully compiled version without erasing any of the work you've done.  It does replace the contents of subfiles, and of the main .tex file.

If you have not altered the aircraft, simply run the Compile script.
\begin{verbatim}
    python .\Compile.py
\end{verbatim}
Once complete, your file should be ready.
\section{Basic \LaTeX}
For all the below, use only what is between the quotes.
\\\\
\setlength{\parindent}{0em}
To make a paragraph break, include an empty line or ``\textbackslash par''.
\\\\
To manually place a line break, for example to keep the table from getting too wide, use ``\textbackslash\textbackslash''.
\\\\
Underline using ``\textbackslash underline\{Text to underline\}''
\\\\
The font used in the plane catalogs does not have a bold face, but italics are ``\textbackslash emph\{Text to italicize\}''
\\\\
Font size can be changed by using ``\textbackslash size'' where the valid sizes are tiny, scriptsize, footnotesize, small, normalsize, large, Large, LARGE, huge, and Huge.
\\\\
Lists are easy, but there are several types.  Check them out \href{https://www.overleaf.com/learn/latex/Lists}{\color{blue}\underline{here.}}

\section{Troubleshooting}

Observed problem: Running the python command just prints ``Python'', and nothing happens.
\\
Solution: Run the python installer, choose modify installation, and ensure the ``Add Python to environment variables'' box is checked, then re-open the PowerShell or cmd window.
\\
If the box is already checked, or if this does not resolve the problem, type ``Get-Command python''  If it shows the existence of a python.exe with version 0.0.0.0, navigate to the location given by Source, and delete it.  This is the Windows Store's attempt to be helpful, but it is failing.

\end{document}